\documentclass[12pt]{article}
%\usepackage{report}

\usepackage[utf8]{inputenc} % allow utf-8 input
\usepackage[T1]{fontenc}    % use 8-bit T1 fonts
\usepackage[colorlinks=true, linkcolor=blue, citecolor=blue, urlcolor=blue]{hyperref}       % hyperlinks
\usepackage{url}            % simple URL typesetting
\usepackage{booktabs}       % professional-quality tables
\usepackage{amsfonts}       % blackboard math symbols
\usepackage{nicefrac}       % compact symbols for 1/2, etc.
%\usepackage{microtype}      % microtypography
\usepackage{lipsum}		% Can be removed after putting your text content
\usepackage{graphicx}
\usepackage{footnote}
\usepackage{doi}
\usepackage{comment}
\usepackage{multirow}
\usepackage{textcomp}
\usepackage{gensymb}
\usepackage{float}
\usepackage{amsmath}
\usepackage{subfigure}
\usepackage{setspace}
\usepackage[skip=10pt plus1pt, indent=30pt]{parskip}
\usepackage[top=1.5in, bottom=1.5in, left=1in, right=1in]{geometry}
\usepackage{titlesec}
\begin{document}


%%%%%%%%%%%%%%%%%%%%%%%%%%%%%%%%%%%%%%%%%%%%%%%%%%%%%%%%%%%%%%%%%%%%%%%%%%%%%
%%%                                                                      %%%
%%%                  ������  !!!  READ ME  !!!  ������                   %%%
%%%                                                                      %%%
%%%    note with your intials if any changes/comments are added! thx     %%%
%%%                                                                      %%%
%%%%%%%%%%%%%%%%%%%%%%%%%%%%%%%%%%%%%%%%%%%%%%%%%%%%%%%%%%%%%%%%%%%%%%%%%%%%%


\begin{titlepage}
    \centering
    \includegraphics[width=2.5cm]{Figures/crest.jpg}\par
    \vspace{0.5cm}
    {\scshape\Large School of Mathematical Sciences \par}
    \vspace{0.25cm}
    {\scshape\Large The University of Southampton \par}
    \vspace{0.25cm}
    {\Large MATH 6149 - Modelling with Differential Equations \par}
    \vspace{1cm}
    {\huge\bfseries On the equations of motion of a swing\par}
    \vspace{1cm}
    {\Large Ben Crossland \par}
    \vspace{0.25cm}
    {\Large Chin Phin Ong (Linus) \par}
    \vspace{0.25cm}
    {\Large Gaby \par} %I dont know your full name, please insert full name here -L
    \vspace{0.25cm}
    {\Large Jacob Smith \par}
    \vspace{0.25cm}
    {\Large Zayn Khan \par}
    \vspace{0.25cm}
    {\large  \par}
    \vfill
    {\large January 2025 \par}
\end{titlepage}

\begin{abstract}
    %insert abstract here -L 
\end{abstract}

\newpage

\section{Introduction}
The aim of this coursework is to investigate the behaviour of a person standing on a swing and moving the swing via an up and down motion of the body on the swing, like in the extreme sport kiiking.

Kiiking (from the Estonian word "kiik", meaning swing) is a sport where the goal is to make a full rotation of the swing (that is attached to the fulcrum via steel beams. The person who successfully does a full rotation with the longest shaft is the winner. The way one would operate a kiiking swing is by "pumping", standing up and squatting down on the swing. The swing gains energy and with the correct technique, one will be able to swing higher and higher, and finally do a full rotation.

In the following sections, we will attempt to model a person on a kiiking swing using differential equations to find the optimal kiiking pattern and discuss the limitations of the model. %-L

\section{On Swings}
%can write up about what steps were taken (like going to an actual swing set) to effectively express the model in equation form) -L

%maybe this section can be combined with the section below (on the model) - thoughts? -L

\section{The Model}
%derivation of the differential equation, and solving, parameter space... -L

\section{Limitations of the Model}
%limitations - e.g. valid for certain angles, works for solid steel swing pole, etc. -L

\section{Conclusion}
%conclusion, futher improvement suggestions - if improvements can be implemented quick and dirty may be worth as a subsection to the limitaitons section. -L

\section{Individual contributions}
\begin{itemize}
    \item Ben Crossland - contributions
    \item Chin Phin Ong (Linus) - contributions 2
    \item Gabby -
    \item Jacob Smith -
    \item Zayn Khan -
\end{itemize}
\end{document}
