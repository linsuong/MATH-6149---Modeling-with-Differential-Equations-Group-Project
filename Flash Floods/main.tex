\documentclass[12pt]{article}
%\usepackage{report}

\usepackage[utf8]{inputenc} % allow utf-8 input
%\usepackage[T1]{fontenc}    % use 8-bit T1 fonts
\usepackage[colorlinks=true, linkcolor=blue, citecolor=blue, urlcolor=blue]{hyperref}       % hyperlinks
\usepackage{url}            % simple URL typesetting
\usepackage{booktabs}       % professional-quality tables
\usepackage{amsfonts}       % blackboard math symbols
\usepackage{nicefrac}       % compact symbols for 1/2, etc.
%\usepackage{microtype}      % microtypography
\usepackage{lipsum}		% Can be removed after putting your text content
\usepackage{graphicx}
\usepackage{footnote}
\usepackage{doi}
\usepackage{comment}
\usepackage{multirow}
\usepackage{textcomp}
\usepackage{gensymb}
\usepackage{float}
\usepackage{amsmath}
%\usepackage{subfigure}
\usepackage{subcaption}
\usepackage{setspace}
\usepackage[skip=10pt plus1pt]{parskip} %I got rid of indent=30pt to make the paragraphs line up nicer - GI
\usepackage[top=3cm, bottom=5cm, left=2.5cm, right=2.5cm]{geometry}

\title{Flash Floods notes dump}
\author{:D}
\begin{document}
\maketitle
\section{description}
You are asked to develop a model for the formation of a flash flood in a watercourse (e.g.
a river, stream, canal). These can be particularly dangerous in desert areas, particularly in
canyons. This is primarily because large amounts of rain can fall within a short period on
poorly drained soil. A surprising feature of such flash floods is that they can occur almost
instantaneously many miles away from the area of the rainfall. You should use your model
to understand how such sudden rises in water level can occur.

You are to write a report (no more than 10 pages long) that describes the mathematical
modelling associated with determining how flash flooding occurs. The usual report format
should be followed, i.e the report should contain a short introduction to the problem and
then proceed to indicate the modelling steps you and your group have undertaken. The
report should summarise the results of your group discussions giving careful attention to the
explanations of the assumptions you have made to create the models, the variables that you
used and the equations you studied. You should give a discussion of the implications of your
modelling results for the physical problem. Include a brief discussion of additional physical
factors you might include to further improve the predictions and how you might put these
into the mathematical model.
\section{week 1}
\begin{itemize}
    \item todo: find l(A) for different cross sections
box canyon, wedge (closer to what a riverbed is), parabola, etc. investigate effect on shape
    \item plot solutions and find shock
\end{itemize}

\section{Approxiations/assumptions}
friction is about 0.1

$\alpha$ is about 2$\%$

Assuming rain distribution is gaussian, is volume of water

We want area the rainfall happened over and how far from the river
\end{document}



