\documentclass[12pt]{article}
%\usepackage{report}

\usepackage[utf8]{inputenc} % allow utf-8 input
%\usepackage[T1]{fontenc}    % use 8-bit T1 fonts
\usepackage[colorlinks=true, linkcolor=blue, citecolor=blue, urlcolor=blue]{hyperref}       % hyperlinks
\usepackage{url}            % simple URL typesetting
\usepackage{booktabs}       % professional-quality tables
\usepackage{amsfonts}       % blackboard math symbols
\usepackage{nicefrac}       % compact symbols for 1/2, etc.
%\usepackage{microtype}      % microtypography
\usepackage{lipsum}		% Can be removed after putting your text content
\usepackage{graphicx}
\usepackage{footnote}
\usepackage{doi}
\usepackage{comment}
\usepackage{multirow}
\usepackage{textcomp}
\usepackage{gensymb}
\usepackage{float}
\usepackage{amsmath}
%\usepackage{subfigure}
\usepackage{subcaption}
\usepackage{dirtree}
\usepackage{setspace}
\usepackage{soul}
\usepackage[skip=10pt plus1pt]{parskip} %I got rid of indent=30pt to make the paragraphs line up nicer - GI
\usepackage[top=3cm, bottom=5cm, left=2.5cm, right=2.5cm]{geometry}

\newcommand{\twoeda}{-2e^{\tilde{T}}\text{Da}}
\newcommand{\eda}{e^{\tilde{T}}\text{Da}}
\newcommand{\da}{\text{Da}}

\title{combustion note dump}
\author{:O}
\begin{document}
\maketitle
\section{description}

Heat Equation:
\begin{align} \label{Heat Equation}
    \rho c_p \frac{\partial T}{\partial t} = \frac{\partial}{\partial x} \left(k \frac{\partial T}{\partial x}\right) + G
\end{align}

Non-dimensional Heat Equation:
\begin{equation} \label{Non-dimensional heat equation}
    \frac{\partial \Tilde{T}}{\partial \Tilde{t}} = \frac{1}{Da}\frac{\partial^2 \Tilde{T}}{\partial \Tilde{x}^2} + e^{\Tilde{T}}
\end{equation}

Damkohler number (confirmed by Toby):
\begin{align} \label{Damkohler Number}
    D_a = \frac{k_0 \Delta H L^2}{kT_0 e^{\beta}} \frac{E_a}{R_gT_0} 
\end{align}


Done:
-steady state solution for 1D flat model \\

$\tilde{x} > 0$
\begin{align} \label{Steady State Solution}
    \frac{1}{\sqrt{c}} \log\left| \frac{\sqrt{-2e^{\tilde{T}}D_a +c}\pm\sqrt{c}}{\sqrt{-2e^{\tilde{T}}D_a +c}\mp\sqrt{c}} \right| = \tilde{x} + a
\end{align}

where the top case applies to $x>0$, and the bottom case applies to $x<0$.


C in terms of $\tilde{T}_0$, where $\tilde{T}_0$ is $\tilde{T}$ at $\tilde{x} = 0$. This would be constant for a steady state, as $\frac{\partial \tilde{T}}{\partial \tilde{t}} = 0$.
\begin{align}
    c = 2e^{\tilde{T}_0}D_a
\end{align}

\begin{align}
    a = \frac{1}{\sqrt{2e^{\tilde{T}_0}D_a}} \log\left| \frac{\sqrt{e^{\tilde{T}_0}-1} \mp \sqrt{e^{\tilde{T}_0}}}{\sqrt{e^{\tilde{T}_0}-1} \pm \sqrt{e^{\tilde{T}_0}}} \right| \mp 1,
\end{align}
where the top case applies to $x>0$, and the bottom case applies to $x<0$.

-"1D" cylinder assuming axisymmetric temperature - look at steady state sols\\


To do:

%%%%% before copying please add the following commads to your preamble (the section before \begin{document}:
%\newcommand{\twoeda}{-2e^{\tilde{T}}\text{Da}}
%\newcommand{\eda}{e^{\tilde{T}}\text{Da}}
%\newcommand{\da}{\text{Da}}

%%%%%%%%%%%%%%
\section{step by step derivation for constants}
\begin{equation}
    \begin{split}
         \frac{d\tilde{T}}{d\tilde{x}} = \pm \sqrt{\twoeda + c}
         \\\implies \int\frac{\pm1}{\sqrt{\twoeda + c}}d\tilde{T} = \int dx
         \\ u = \sqrt{\twoeda + c}
         \\ \frac{du}{d\tilde{T}} = \frac{1}{2}\left (\twoeda + c\right) ^ {-\frac{1}{2}}\cdot \left(\twoeda\right)
         \\ =\frac{-e^{\tilde{T}}\text{Da}}{\sqrt{\twoeda + c}} 
         \\ \implies d\tilde{T} = \frac{\sqrt{\twoeda}}{-e^{\tilde{T}}\text{Da}}du
         \\ \implies \int \frac{\pm 1}{e^{\tilde{T}}\text{Da}}du = \int dx
         \\ \frac{u^2 - c}{2} = -e^{\tilde{T}}\text{Da}
         \implies \int \frac{\pm 2}{u^2 - c}du = \int dx
         \\ \implies \int \frac{\pm 2}{(u + \sqrt{c})(u - \sqrt{c}}du = \int dx
    \end{split}
\end{equation}
part 2: 
\begin{equation}
    \begin{split}
         \int \frac{\pm 2}{(u + \sqrt{c})(u - \sqrt{c}}du = \frac{1}{\sqrt{c}}\left[\ln{|u - \sqrt{c}| - \ln|u + \sqrt{c}|} \right]\\ = \frac{1}{\sqrt{c}}\ln\left|\frac{u \mp c}{u \pm c} \right|
        \\ \frac{1}{\sqrt{c}}\ln\left|\frac{\sqrt{\twoeda + c} \mp c}{\sqrt{\twoeda + c} \pm c} \right|
        \\\implies \frac{1}{\sqrt{c}}\ln\left|\frac{\sqrt{\twoeda + c} \mp \sqrt{c}}{\sqrt{\twoeda + c} \pm \sqrt{c}} \right| = x + a
        \\ \implies \frac{(\sqrt{v}\mp \sqrt{c})(\sqrt{c - 2\da}\pm \sqrt{c})}{(\sqrt{v}\pm \sqrt{c})(\sqrt{c - 2\da}) \mp \sqrt{c}} = e^{\pm \sqrt{c}}
    \end{split}
\end{equation}
The boundary condition that $\tilde{T} = 0$, at $x = \pm 1$ gives us:
\begin{equation}
    a = \frac{1}{\sqrt{c}}\ln\left|\frac{\sqrt{-2Da + c} \mp \sqrt{c}}{\sqrt{-2Da + c} \pm \sqrt{c}} \right| \pm 1,
\end{equation}
we then get: 
\begin{equation}
    \frac{1}{\sqrt{c}}\ln\left|\frac{\sqrt{\twoeda + c} \mp \sqrt{c}}{\sqrt{\twoeda + c} \pm \sqrt{c}} \right| = x + \frac{1}{\sqrt{c}}\ln\left|\frac{\sqrt{-2Da + c} \mp \sqrt{c}}{\sqrt{-2Da + c} \pm \sqrt{c}} \right| \pm 1.
\end{equation}
From before, we had:
\begin{equation}
    \frac{d\tilde{T}}{d\tilde{x}} = \pm \sqrt{\twoeda + c},
\end{equation}
and then defining $v = -2e^{\tilde{T}}Da +c$ we have:
\begin{equation}
    \frac{d\tilde{T}}{d\tilde{x}} = \pm \sqrt{v},
\end{equation}
which is equal to $0$ when $x = 0$ due to the boundary condition that $\frac{\partial\tilde{T}}{\partial \tilde{x}} = 0$ at $x = 0$. 
Also at $x = 0$ we get:
\begin{equation}
    \frac{1}{\sqrt{c}}\ln\left|\frac{\sqrt{v} \mp \sqrt{c}}{\sqrt{v} \pm \sqrt{c}} \right| = \frac{1}{\sqrt{c}}\ln\left|\frac{\sqrt{-2Da + c} \mp \sqrt{c}}{\sqrt{-2Da + c} \pm \sqrt{c}} \right| \pm 1.
\end{equation}
part 3
\begin{equation}
\begin{split}
\frac{(\sqrt{v}\mp \sqrt{c})(\sqrt{c - 2\da}\pm \sqrt{c})}{(\sqrt{v}\pm \sqrt{c})(\sqrt{c - 2\da}) \mp \sqrt{c}} = e^{\pm \sqrt{c}}\\
\implies
    \sqrt{v}\sqrt{c - 2\da}\pm\sqrt{v}\sqrt{c}-e^{\pm\sqrt{c}}[\sqrt{v}\sqrt{c - 2\da} \mp \sqrt{v}\sqrt{c}] = e^{\pm \sqrt{c}}\left[\pm \sqrt{c^2 - 2c \da} - c\right] \pm \sqrt{c^2 - 2c\da} + c
    \\ \sqrt{v} = \frac{e^{\pm \sqrt{c}}\left[\pm \sqrt{c^2 - 2c \da} - c\right] \pm \sqrt{c^2 - 2c\da} + c}{\sqrt{c - 2\da}\pm\sqrt{c}-e^{\pm\sqrt{c}}[\sqrt{c - 2\da} \mp \sqrt{c}]}
\end{split}
\end{equation}
$\sqrt{v} = \frac{d\tilde{T}}{d\tilde{x}}$ at $x = 0$:

\begin{equation}
    \begin{split}
        e^{\pm \sqrt{c}}\left[\pm\sqrt{c^2 - 2\da} - c\right] \pm \sqrt{c^2 - 2c\da} + c = 0
        \\ \text{c}\neq 0:
        \\ \implies e^{\pm\sqrt{c}} = \frac{\mp \sqrt{c - 2\da} - \sqrt{c}}{\pm \sqrt{c - 2\da} - \sqrt{c}}
        \\ \implies \sqrt{c} = \ln\left| \frac{\pm \sqrt{c - 2\da} - \sqrt{c}}{\mp \sqrt{c - 2\da} - \sqrt{c}}\right|.
    \end{split}
\end{equation}



\section{some long equations i typed out}
Eqn 2 in method of lines:
\begin{equation}
    \frac{d\vec{u}}{dt} = \frac{1}{\Delta x^2}
        \begin{pmatrix}
            -2 & 1& \cdots & 0 \\
            1 & -2 & 1 & \\
            0&\ddots & \ddots & \ddots\\
            \vdots&1 & -2 & 1 &\\\
            0 & \cdots &1 &-2 
        \end{pmatrix}
        \vec{u}(t) + \frac{1}{\Delta x^2}
        \begin{pmatrix}
            A(t) \\
            0 \\
            \vdots \\
            0 \\
            B(t)
        \end{pmatrix}
        + \begin{pmatrix}
            S(u_1, x_1)\\
            \vdots \\
            S(u_{N-1}, x_{N-1})
        
        \end{pmatrix}
\end{equation}
\section{Real world quantities}
Coal:
$\rho = 1300 kg/m^3$, $c_p = 1262$, $\kappa = 0.2$, $E_a = 396.52, $\Delta H = 120 kJ/kg$, $k_0$

Sawdust:
$\rho = 210 kg/m^3$, $c_p = 0.9$, $\kappa = 0.08 J/smK$, $E_a = 269 kJ/mol$, $\Delta H = $, $k_0 = $

Pistachios:
$\rho \approx 500 kg/m^3$, $c_p = 1.6 kJ/kg$, $\kappa$, $E_a = 8.75 J/mol$, $\Delta H$, $k_0$
\end{document}