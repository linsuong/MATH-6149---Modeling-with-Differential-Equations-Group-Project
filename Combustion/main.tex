\documentclass[12pt]{article}
%\usepackage{report}

\usepackage[utf8]{inputenc} % allow utf-8 input
%\usepackage[T1]{fontenc}    % use 8-bit T1 fonts
\usepackage[colorlinks=true, linkcolor=blue, citecolor=blue, urlcolor=blue]{hyperref}       % hyperlinks
\usepackage{url}            % simple URL typesetting
\usepackage{booktabs}       % professional-quality tables
\usepackage{amsfonts}       % blackboard math symbols
\usepackage{nicefrac}       % compact symbols for 1/2, etc.
%\usepackage{microtype}      % microtypography
\usepackage{lipsum}		% Can be removed after putting your text content
\usepackage{graphicx}
\usepackage{footnote}
\usepackage{doi}
\usepackage{comment}
\usepackage{multirow}
\usepackage{textcomp}
\usepackage{gensymb}
\usepackage{float}
\usepackage{amsmath}
%\usepackage{subfigure}
\usepackage{subcaption}
\usepackage{dirtree}
\usepackage{setspace}
\usepackage{soul}
\usepackage[skip=10pt plus1pt]{parskip} %I got rid of indent=30pt to make the paragraphs line up nicer - GI
\usepackage[top=3cm, bottom=5cm, left=2.5cm, right=2.5cm]{geometry}

\title{combustion note dump}
\author{:O}
\begin{document}
\maketitle
\section{description}

Heat Equation:
\begin{align} \label{Heat Equation}
    \rho c_p \frac{\partial T}{\partial t} = \frac{\partial}{\partial x} \left(k \frac{\partial T}{\partial x}\right) + G
\end{align}

Non-dimensional Heat Equation:
\begin{equation} \label{Non-dimensional heat equation}
    \frac{\partial \Tilde{T}}{\partial \Tilde{t}} = \frac{1}{Da}\frac{\partial^2 \Tilde{T}}{\partial \Tilde{x}^2} + e^{\Tilde{T}}
\end{equation}

Damkohler number (confirmed by Toby):
\begin{align} \label{Damkohler Number}
    D_a = \frac{k_0 \Delta H L^2}{kT_0 e^{\beta}} \frac{E_a}{R_gT_0} 
\end{align}


Done:
-steady state solution for 1D flat model; conclusion (temporary): no steady state soln as c=0; maybe do limit? \\

$\tilde{x} > 0$
\begin{align} \label{Steady State Solution}
    \frac{1}{\sqrt{c}} \log\left| \frac{\sqrt{-2e^{\tilde{T}}D_a +c}\pm\sqrt{c}}{\sqrt{-2e^{\tilde{T}}D_a +c}\mp\sqrt{c}} \right| = \tilde{x} + a
\end{align}

where the top case applies to $x>0$, and the bottom case applies to $x<0$.


C in terms of $\tilde{T}_0$, where $\tilde{T}_0$ is $\tilde{T}$ at $\tilde{x} = 0$. This would be constant for a steady state, as $\frac{\partial \tilde{T}}{\partial \tilde{t}} = 0$.
\begin{align}
    c = 2e^{\tilde{T}_0}D_a
\end{align}

\begin{align}
    a = \frac{1}{\sqrt{2e^{\tilde{T}_0}D_a}} \log\left| \frac{\sqrt{e^{\tilde{T}_0}-1} \mp \sqrt{e^{\tilde{T}_0}}}{\sqrt{e^{\tilde{T}_0}-1} \pm \sqrt{e^{\tilde{T}_0}}} \right| \mp 1,
\end{align}
where the top case applies to $x>0$, and the bottom case applies to $x<0$.

-"1D" cylinder assuming axisymmetric temperature - look at steady state sols\\


To do:


\end{document}